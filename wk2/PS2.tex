\documentclass[11pt]{article}


% Useful Packages

\usepackage{latexsym}
\usepackage{amssymb}
\usepackage{amsthm}
\usepackage{amsmath}
\usepackage{tikz}
\usepackage{sudoku}
\usepackage{graphicx}

\usepgflibrary{arrows}



% Statement styles

\newtheorem{theorem}{Theorem}[section]
\newtheorem{fact}[theorem]{Fact}
\newtheorem{claim}[theorem]{Claim}
\newtheorem{lemma}[theorem]{Lemma}
\newtheorem{definition}[theorem]{Definition}
\newtheorem{proposition}[theorem]{Proposition}
\newtheorem{corollary}[theorem]{Corollary}
\newtheorem{conjecture}[theorem]{Conjecture}

\theoremstyle{definition}
\newtheorem*{example}{Example}
\newtheorem*{remark}{Remark}
\newtheorem*{remarks}{Remarks}

\numberwithin{equation}{section}


% Page dimensions

\setlength{\evensidemargin}{1in}
\addtolength{\evensidemargin}{-1in}
\setlength{\oddsidemargin}{1in}
\addtolength{\oddsidemargin}{-1in}
\setlength{\topmargin}{1in}
\addtolength{\topmargin}{-1.5in}

\setlength{\textwidth}{16.5cm}
\setlength{\textheight}{23cm}


%Special Letters

\newcommand{\CC}{\mathbb{C}} 
\newcommand{\FF}{\mathbb{F}}  
\newcommand{\KK}{\mathbb{K}}
\newcommand{\QQ}{\mathbb{Q}}  
\newcommand{\ZZ}{\mathbb{Z}}

\newcommand{\cA}{\mathcal{A}}
\newcommand{\cB}{\mathcal{B}} 
\newcommand{\cC}{\mathcal{C}}
\newcommand{\cD}{\mathcal{D}} 
\newcommand{\cE}{\mathcal{E}} 
\newcommand{\cF}{\mathcal{F}}
\newcommand{\cG}{\mathcal{G}}  
\newcommand{\cH}{\mathcal{H}}
\newcommand{\cI}{\mathcal{I}}
\newcommand{\cJ}{\mathcal{J}}
\newcommand{\cK}{\mathcal{K}}
\newcommand{\cL}{\mathcal{L}}
\newcommand{\cM}{\mathcal{M}}
\newcommand{\cN}{\mathcal{N}}
\newcommand{\cO}{\mathcal{O}}
\newcommand{\cP}{\mathcal{P}}
\newcommand{\cQ}{\mathcal{Q}}
\newcommand{\cR}{\mathcal{R}}
\newcommand{\cS}{\mathcal{S}}
\newcommand{\cT}{\mathcal{T}}
\newcommand{\cU}{\mathcal{U}}
\newcommand{\cV}{\mathcal{V}}
\newcommand{\cW}{\mathcal{W}}
\newcommand{\cX}{\mathcal{X}}
\newcommand{\cY}{\mathcal{Y}}
\newcommand{\cZ}{\mathcal{Z}}

\newcommand{\fkg}{\mathfrak{g}}
\newcommand{\fkh}{\mathfrak{h}}
\newcommand{\fkn}{\mathfrak{n}}
\newcommand{\fksl}{\mathfrak{sl}}
\newcommand{\fku}{\mathfrak{u}}
\newcommand{\fkS}{\mathfrak{S}}
\newcommand{\fkX}{\mathfrak{X}}

\newcommand{\bg}{\mathbf{g}}
\newcommand{\bk}{\mathbf{k}}
\newcommand{\bm}{\mathbf{m}}
\newcommand{\bM}{\mathbf{M}}
\newcommand{\bN}{\mathbf{N}}
\newcommand{\bp}{\mathbf{p}}




\allowdisplaybreaks[1]

\begin{document}

\title{Problem Set 2}
\author{Max Harris}

\maketitle

Collaborators: Dennis (has a beard, sits next to me)
\begin{enumerate}
	\item (20 pts) Suppose list a has $n$ elements and is sorted.
	
	\begin{enumerate}
		\item (10 pts) Using $\Theta$ notation, what is the best case running time as function of $n$?\\
		
		The best case running time is when we enter the \emph{while} loop and return on the first iteration. In order to return on the first iteration, the \emph{target} we are looking for has to be in the middle of the sorted list so that 
		$$m = (x+y)/2 = \text{the position of the target element}$$ 
		and thus 
		$$a[m] = target.$$ 
		After entering the \emph{while} loop for the first time, $m$ is assigned to the index of the middle element. If $a[m] = target$ it will return on the first iteration. This is our best case. Since the best case doesn't depend on $n$, it will run in constant time or in $\Theta$ notation as a function of n:
		$$f(n) = \Theta(g(n)) = \Theta(n^0) = \Theta(1).$$
		The notation $\Theta(n^0)$ is beneficial because it explicitly expresses which variable is tending toward infinity.\\
		There are some operations that occur even in our best case scenario but the number of operations will always be constant. If we let $C_0$ be the number of constant operations performed than we can verify that there exists $n_0$ and constants $c_1 < C_0$ and  $c_2 > C_0$ such that $c_1g(n) \leq f(n) \leq c_2g(n)$ for all $n > n_0$. \\
		
		\item Using $\Theta$ notation, what is the worst case running time as function of n?\\
		
		Since this is a search algorithm, the worst case is when the list does not contain the $target$
so we search the entire list. The \emph{while} loop dictates how many iterations $i$ are executed. 

	\end{enumerate}	
	
\end{enumerate}








\end{document}